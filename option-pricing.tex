\documentclass[11pt]{article}

    \usepackage[breakable]{tcolorbox}
    \usepackage{parskip} % Stop auto-indenting (to mimic markdown behaviour)
    
    \usepackage{iftex}
    \ifPDFTeX
    	\usepackage[T1]{fontenc}
    	\usepackage{mathpazo}
    \else
    	\usepackage{fontspec}
    \fi

    % Basic figure setup, for now with no caption control since it's done
    % automatically by Pandoc (which extracts ![](path) syntax from Markdown).
    \usepackage{graphicx}
    % Maintain compatibility with old templates. Remove in nbconvert 6.0
    \let\Oldincludegraphics\includegraphics
    % Ensure that by default, figures have no caption (until we provide a
    % proper Figure object with a Caption API and a way to capture that
    % in the conversion process - todo).
    \usepackage{caption}
    \DeclareCaptionFormat{nocaption}{}
    \captionsetup{format=nocaption,aboveskip=0pt,belowskip=0pt}

    \usepackage[Export]{adjustbox} % Used to constrain images to a maximum size
    \adjustboxset{max size={0.9\linewidth}{0.9\paperheight}}
    \usepackage{float}
    \floatplacement{figure}{H} % forces figures to be placed at the correct location
    \usepackage{xcolor} % Allow colors to be defined
    \usepackage{enumerate} % Needed for markdown enumerations to work
    \usepackage{geometry} % Used to adjust the document margins
    \usepackage{amsmath} % Equations
    \usepackage{amssymb} % Equations
    \usepackage{textcomp} % defines textquotesingle
    % Hack from http://tex.stackexchange.com/a/47451/13684:
    \AtBeginDocument{%
        \def\PYZsq{\textquotesingle}% Upright quotes in Pygmentized code
    }
    \usepackage{upquote} % Upright quotes for verbatim code
    \usepackage{eurosym} % defines \euro
    \usepackage[mathletters]{ucs} % Extended unicode (utf-8) support
    \usepackage{fancyvrb} % verbatim replacement that allows latex
    \usepackage{grffile} % extends the file name processing of package graphics 
                         % to support a larger range
    \makeatletter % fix for grffile with XeLaTeX
    \def\Gread@@xetex#1{%
      \IfFileExists{"\Gin@base".bb}%
      {\Gread@eps{\Gin@base.bb}}%
      {\Gread@@xetex@aux#1}%
    }
    \makeatother

    % The hyperref package gives us a pdf with properly built
    % internal navigation ('pdf bookmarks' for the table of contents,
    % internal cross-reference links, web links for URLs, etc.)
    \usepackage{hyperref}
    % The default LaTeX title has an obnoxious amount of whitespace. By default,
    % titling removes some of it. It also provides customization options.
    \usepackage{titling}
    \usepackage{longtable} % longtable support required by pandoc >1.10
    \usepackage{booktabs}  % table support for pandoc > 1.12.2
    \usepackage[inline]{enumitem} % IRkernel/repr support (it uses the enumerate* environment)
    \usepackage[normalem]{ulem} % ulem is needed to support strikethroughs (\sout)
                                % normalem makes italics be italics, not underlines
    \usepackage{mathrsfs}
    

    
    % Colors for the hyperref package
    \definecolor{urlcolor}{rgb}{0,.145,.698}
    \definecolor{linkcolor}{rgb}{.71,0.21,0.01}
    \definecolor{citecolor}{rgb}{.12,.54,.11}

    % ANSI colors
    \definecolor{ansi-black}{HTML}{3E424D}
    \definecolor{ansi-black-intense}{HTML}{282C36}
    \definecolor{ansi-red}{HTML}{E75C58}
    \definecolor{ansi-red-intense}{HTML}{B22B31}
    \definecolor{ansi-green}{HTML}{00A250}
    \definecolor{ansi-green-intense}{HTML}{007427}
    \definecolor{ansi-yellow}{HTML}{DDB62B}
    \definecolor{ansi-yellow-intense}{HTML}{B27D12}
    \definecolor{ansi-blue}{HTML}{208FFB}
    \definecolor{ansi-blue-intense}{HTML}{0065CA}
    \definecolor{ansi-magenta}{HTML}{D160C4}
    \definecolor{ansi-magenta-intense}{HTML}{A03196}
    \definecolor{ansi-cyan}{HTML}{60C6C8}
    \definecolor{ansi-cyan-intense}{HTML}{258F8F}
    \definecolor{ansi-white}{HTML}{C5C1B4}
    \definecolor{ansi-white-intense}{HTML}{A1A6B2}
    \definecolor{ansi-default-inverse-fg}{HTML}{FFFFFF}
    \definecolor{ansi-default-inverse-bg}{HTML}{000000}

    % commands and environments needed by pandoc snippets
    % extracted from the output of `pandoc -s`
    \providecommand{\tightlist}{%
      \setlength{\itemsep}{0pt}\setlength{\parskip}{0pt}}
    \DefineVerbatimEnvironment{Highlighting}{Verbatim}{commandchars=\\\{\}}
    % Add ',fontsize=\small' for more characters per line
    \newenvironment{Shaded}{}{}
    \newcommand{\KeywordTok}[1]{\textcolor[rgb]{0.00,0.44,0.13}{\textbf{{#1}}}}
    \newcommand{\DataTypeTok}[1]{\textcolor[rgb]{0.56,0.13,0.00}{{#1}}}
    \newcommand{\DecValTok}[1]{\textcolor[rgb]{0.25,0.63,0.44}{{#1}}}
    \newcommand{\BaseNTok}[1]{\textcolor[rgb]{0.25,0.63,0.44}{{#1}}}
    \newcommand{\FloatTok}[1]{\textcolor[rgb]{0.25,0.63,0.44}{{#1}}}
    \newcommand{\CharTok}[1]{\textcolor[rgb]{0.25,0.44,0.63}{{#1}}}
    \newcommand{\StringTok}[1]{\textcolor[rgb]{0.25,0.44,0.63}{{#1}}}
    \newcommand{\CommentTok}[1]{\textcolor[rgb]{0.38,0.63,0.69}{\textit{{#1}}}}
    \newcommand{\OtherTok}[1]{\textcolor[rgb]{0.00,0.44,0.13}{{#1}}}
    \newcommand{\AlertTok}[1]{\textcolor[rgb]{1.00,0.00,0.00}{\textbf{{#1}}}}
    \newcommand{\FunctionTok}[1]{\textcolor[rgb]{0.02,0.16,0.49}{{#1}}}
    \newcommand{\RegionMarkerTok}[1]{{#1}}
    \newcommand{\ErrorTok}[1]{\textcolor[rgb]{1.00,0.00,0.00}{\textbf{{#1}}}}
    \newcommand{\NormalTok}[1]{{#1}}
    
    % Additional commands for more recent versions of Pandoc
    \newcommand{\ConstantTok}[1]{\textcolor[rgb]{0.53,0.00,0.00}{{#1}}}
    \newcommand{\SpecialCharTok}[1]{\textcolor[rgb]{0.25,0.44,0.63}{{#1}}}
    \newcommand{\VerbatimStringTok}[1]{\textcolor[rgb]{0.25,0.44,0.63}{{#1}}}
    \newcommand{\SpecialStringTok}[1]{\textcolor[rgb]{0.73,0.40,0.53}{{#1}}}
    \newcommand{\ImportTok}[1]{{#1}}
    \newcommand{\DocumentationTok}[1]{\textcolor[rgb]{0.73,0.13,0.13}{\textit{{#1}}}}
    \newcommand{\AnnotationTok}[1]{\textcolor[rgb]{0.38,0.63,0.69}{\textbf{\textit{{#1}}}}}
    \newcommand{\CommentVarTok}[1]{\textcolor[rgb]{0.38,0.63,0.69}{\textbf{\textit{{#1}}}}}
    \newcommand{\VariableTok}[1]{\textcolor[rgb]{0.10,0.09,0.49}{{#1}}}
    \newcommand{\ControlFlowTok}[1]{\textcolor[rgb]{0.00,0.44,0.13}{\textbf{{#1}}}}
    \newcommand{\OperatorTok}[1]{\textcolor[rgb]{0.40,0.40,0.40}{{#1}}}
    \newcommand{\BuiltInTok}[1]{{#1}}
    \newcommand{\ExtensionTok}[1]{{#1}}
    \newcommand{\PreprocessorTok}[1]{\textcolor[rgb]{0.74,0.48,0.00}{{#1}}}
    \newcommand{\AttributeTok}[1]{\textcolor[rgb]{0.49,0.56,0.16}{{#1}}}
    \newcommand{\InformationTok}[1]{\textcolor[rgb]{0.38,0.63,0.69}{\textbf{\textit{{#1}}}}}
    \newcommand{\WarningTok}[1]{\textcolor[rgb]{0.38,0.63,0.69}{\textbf{\textit{{#1}}}}}
    
    
    % Define a nice break command that doesn't care if a line doesn't already
    % exist.
    \def\br{\hspace*{\fill} \\* }
    % Math Jax compatibility definitions
    \def\gt{>}
    \def\lt{<}
    \let\Oldtex\TeX
    \let\Oldlatex\LaTeX
    \renewcommand{\TeX}{\textrm{\Oldtex}}
    \renewcommand{\LaTeX}{\textrm{\Oldlatex}}
    % Document parameters
    % Document title
    \title{option-pricing}
    
    
    
    
    
% Pygments definitions
\makeatletter
\def\PY@reset{\let\PY@it=\relax \let\PY@bf=\relax%
    \let\PY@ul=\relax \let\PY@tc=\relax%
    \let\PY@bc=\relax \let\PY@ff=\relax}
\def\PY@tok#1{\csname PY@tok@#1\endcsname}
\def\PY@toks#1+{\ifx\relax#1\empty\else%
    \PY@tok{#1}\expandafter\PY@toks\fi}
\def\PY@do#1{\PY@bc{\PY@tc{\PY@ul{%
    \PY@it{\PY@bf{\PY@ff{#1}}}}}}}
\def\PY#1#2{\PY@reset\PY@toks#1+\relax+\PY@do{#2}}

\expandafter\def\csname PY@tok@w\endcsname{\def\PY@tc##1{\textcolor[rgb]{0.73,0.73,0.73}{##1}}}
\expandafter\def\csname PY@tok@c\endcsname{\let\PY@it=\textit\def\PY@tc##1{\textcolor[rgb]{0.25,0.50,0.50}{##1}}}
\expandafter\def\csname PY@tok@cp\endcsname{\def\PY@tc##1{\textcolor[rgb]{0.74,0.48,0.00}{##1}}}
\expandafter\def\csname PY@tok@k\endcsname{\let\PY@bf=\textbf\def\PY@tc##1{\textcolor[rgb]{0.00,0.50,0.00}{##1}}}
\expandafter\def\csname PY@tok@kp\endcsname{\def\PY@tc##1{\textcolor[rgb]{0.00,0.50,0.00}{##1}}}
\expandafter\def\csname PY@tok@kt\endcsname{\def\PY@tc##1{\textcolor[rgb]{0.69,0.00,0.25}{##1}}}
\expandafter\def\csname PY@tok@o\endcsname{\def\PY@tc##1{\textcolor[rgb]{0.40,0.40,0.40}{##1}}}
\expandafter\def\csname PY@tok@ow\endcsname{\let\PY@bf=\textbf\def\PY@tc##1{\textcolor[rgb]{0.67,0.13,1.00}{##1}}}
\expandafter\def\csname PY@tok@nb\endcsname{\def\PY@tc##1{\textcolor[rgb]{0.00,0.50,0.00}{##1}}}
\expandafter\def\csname PY@tok@nf\endcsname{\def\PY@tc##1{\textcolor[rgb]{0.00,0.00,1.00}{##1}}}
\expandafter\def\csname PY@tok@nc\endcsname{\let\PY@bf=\textbf\def\PY@tc##1{\textcolor[rgb]{0.00,0.00,1.00}{##1}}}
\expandafter\def\csname PY@tok@nn\endcsname{\let\PY@bf=\textbf\def\PY@tc##1{\textcolor[rgb]{0.00,0.00,1.00}{##1}}}
\expandafter\def\csname PY@tok@ne\endcsname{\let\PY@bf=\textbf\def\PY@tc##1{\textcolor[rgb]{0.82,0.25,0.23}{##1}}}
\expandafter\def\csname PY@tok@nv\endcsname{\def\PY@tc##1{\textcolor[rgb]{0.10,0.09,0.49}{##1}}}
\expandafter\def\csname PY@tok@no\endcsname{\def\PY@tc##1{\textcolor[rgb]{0.53,0.00,0.00}{##1}}}
\expandafter\def\csname PY@tok@nl\endcsname{\def\PY@tc##1{\textcolor[rgb]{0.63,0.63,0.00}{##1}}}
\expandafter\def\csname PY@tok@ni\endcsname{\let\PY@bf=\textbf\def\PY@tc##1{\textcolor[rgb]{0.60,0.60,0.60}{##1}}}
\expandafter\def\csname PY@tok@na\endcsname{\def\PY@tc##1{\textcolor[rgb]{0.49,0.56,0.16}{##1}}}
\expandafter\def\csname PY@tok@nt\endcsname{\let\PY@bf=\textbf\def\PY@tc##1{\textcolor[rgb]{0.00,0.50,0.00}{##1}}}
\expandafter\def\csname PY@tok@nd\endcsname{\def\PY@tc##1{\textcolor[rgb]{0.67,0.13,1.00}{##1}}}
\expandafter\def\csname PY@tok@s\endcsname{\def\PY@tc##1{\textcolor[rgb]{0.73,0.13,0.13}{##1}}}
\expandafter\def\csname PY@tok@sd\endcsname{\let\PY@it=\textit\def\PY@tc##1{\textcolor[rgb]{0.73,0.13,0.13}{##1}}}
\expandafter\def\csname PY@tok@si\endcsname{\let\PY@bf=\textbf\def\PY@tc##1{\textcolor[rgb]{0.73,0.40,0.53}{##1}}}
\expandafter\def\csname PY@tok@se\endcsname{\let\PY@bf=\textbf\def\PY@tc##1{\textcolor[rgb]{0.73,0.40,0.13}{##1}}}
\expandafter\def\csname PY@tok@sr\endcsname{\def\PY@tc##1{\textcolor[rgb]{0.73,0.40,0.53}{##1}}}
\expandafter\def\csname PY@tok@ss\endcsname{\def\PY@tc##1{\textcolor[rgb]{0.10,0.09,0.49}{##1}}}
\expandafter\def\csname PY@tok@sx\endcsname{\def\PY@tc##1{\textcolor[rgb]{0.00,0.50,0.00}{##1}}}
\expandafter\def\csname PY@tok@m\endcsname{\def\PY@tc##1{\textcolor[rgb]{0.40,0.40,0.40}{##1}}}
\expandafter\def\csname PY@tok@gh\endcsname{\let\PY@bf=\textbf\def\PY@tc##1{\textcolor[rgb]{0.00,0.00,0.50}{##1}}}
\expandafter\def\csname PY@tok@gu\endcsname{\let\PY@bf=\textbf\def\PY@tc##1{\textcolor[rgb]{0.50,0.00,0.50}{##1}}}
\expandafter\def\csname PY@tok@gd\endcsname{\def\PY@tc##1{\textcolor[rgb]{0.63,0.00,0.00}{##1}}}
\expandafter\def\csname PY@tok@gi\endcsname{\def\PY@tc##1{\textcolor[rgb]{0.00,0.63,0.00}{##1}}}
\expandafter\def\csname PY@tok@gr\endcsname{\def\PY@tc##1{\textcolor[rgb]{1.00,0.00,0.00}{##1}}}
\expandafter\def\csname PY@tok@ge\endcsname{\let\PY@it=\textit}
\expandafter\def\csname PY@tok@gs\endcsname{\let\PY@bf=\textbf}
\expandafter\def\csname PY@tok@gp\endcsname{\let\PY@bf=\textbf\def\PY@tc##1{\textcolor[rgb]{0.00,0.00,0.50}{##1}}}
\expandafter\def\csname PY@tok@go\endcsname{\def\PY@tc##1{\textcolor[rgb]{0.53,0.53,0.53}{##1}}}
\expandafter\def\csname PY@tok@gt\endcsname{\def\PY@tc##1{\textcolor[rgb]{0.00,0.27,0.87}{##1}}}
\expandafter\def\csname PY@tok@err\endcsname{\def\PY@bc##1{\setlength{\fboxsep}{0pt}\fcolorbox[rgb]{1.00,0.00,0.00}{1,1,1}{\strut ##1}}}
\expandafter\def\csname PY@tok@kc\endcsname{\let\PY@bf=\textbf\def\PY@tc##1{\textcolor[rgb]{0.00,0.50,0.00}{##1}}}
\expandafter\def\csname PY@tok@kd\endcsname{\let\PY@bf=\textbf\def\PY@tc##1{\textcolor[rgb]{0.00,0.50,0.00}{##1}}}
\expandafter\def\csname PY@tok@kn\endcsname{\let\PY@bf=\textbf\def\PY@tc##1{\textcolor[rgb]{0.00,0.50,0.00}{##1}}}
\expandafter\def\csname PY@tok@kr\endcsname{\let\PY@bf=\textbf\def\PY@tc##1{\textcolor[rgb]{0.00,0.50,0.00}{##1}}}
\expandafter\def\csname PY@tok@bp\endcsname{\def\PY@tc##1{\textcolor[rgb]{0.00,0.50,0.00}{##1}}}
\expandafter\def\csname PY@tok@fm\endcsname{\def\PY@tc##1{\textcolor[rgb]{0.00,0.00,1.00}{##1}}}
\expandafter\def\csname PY@tok@vc\endcsname{\def\PY@tc##1{\textcolor[rgb]{0.10,0.09,0.49}{##1}}}
\expandafter\def\csname PY@tok@vg\endcsname{\def\PY@tc##1{\textcolor[rgb]{0.10,0.09,0.49}{##1}}}
\expandafter\def\csname PY@tok@vi\endcsname{\def\PY@tc##1{\textcolor[rgb]{0.10,0.09,0.49}{##1}}}
\expandafter\def\csname PY@tok@vm\endcsname{\def\PY@tc##1{\textcolor[rgb]{0.10,0.09,0.49}{##1}}}
\expandafter\def\csname PY@tok@sa\endcsname{\def\PY@tc##1{\textcolor[rgb]{0.73,0.13,0.13}{##1}}}
\expandafter\def\csname PY@tok@sb\endcsname{\def\PY@tc##1{\textcolor[rgb]{0.73,0.13,0.13}{##1}}}
\expandafter\def\csname PY@tok@sc\endcsname{\def\PY@tc##1{\textcolor[rgb]{0.73,0.13,0.13}{##1}}}
\expandafter\def\csname PY@tok@dl\endcsname{\def\PY@tc##1{\textcolor[rgb]{0.73,0.13,0.13}{##1}}}
\expandafter\def\csname PY@tok@s2\endcsname{\def\PY@tc##1{\textcolor[rgb]{0.73,0.13,0.13}{##1}}}
\expandafter\def\csname PY@tok@sh\endcsname{\def\PY@tc##1{\textcolor[rgb]{0.73,0.13,0.13}{##1}}}
\expandafter\def\csname PY@tok@s1\endcsname{\def\PY@tc##1{\textcolor[rgb]{0.73,0.13,0.13}{##1}}}
\expandafter\def\csname PY@tok@mb\endcsname{\def\PY@tc##1{\textcolor[rgb]{0.40,0.40,0.40}{##1}}}
\expandafter\def\csname PY@tok@mf\endcsname{\def\PY@tc##1{\textcolor[rgb]{0.40,0.40,0.40}{##1}}}
\expandafter\def\csname PY@tok@mh\endcsname{\def\PY@tc##1{\textcolor[rgb]{0.40,0.40,0.40}{##1}}}
\expandafter\def\csname PY@tok@mi\endcsname{\def\PY@tc##1{\textcolor[rgb]{0.40,0.40,0.40}{##1}}}
\expandafter\def\csname PY@tok@il\endcsname{\def\PY@tc##1{\textcolor[rgb]{0.40,0.40,0.40}{##1}}}
\expandafter\def\csname PY@tok@mo\endcsname{\def\PY@tc##1{\textcolor[rgb]{0.40,0.40,0.40}{##1}}}
\expandafter\def\csname PY@tok@ch\endcsname{\let\PY@it=\textit\def\PY@tc##1{\textcolor[rgb]{0.25,0.50,0.50}{##1}}}
\expandafter\def\csname PY@tok@cm\endcsname{\let\PY@it=\textit\def\PY@tc##1{\textcolor[rgb]{0.25,0.50,0.50}{##1}}}
\expandafter\def\csname PY@tok@cpf\endcsname{\let\PY@it=\textit\def\PY@tc##1{\textcolor[rgb]{0.25,0.50,0.50}{##1}}}
\expandafter\def\csname PY@tok@c1\endcsname{\let\PY@it=\textit\def\PY@tc##1{\textcolor[rgb]{0.25,0.50,0.50}{##1}}}
\expandafter\def\csname PY@tok@cs\endcsname{\let\PY@it=\textit\def\PY@tc##1{\textcolor[rgb]{0.25,0.50,0.50}{##1}}}

\def\PYZbs{\char`\\}
\def\PYZus{\char`\_}
\def\PYZob{\char`\{}
\def\PYZcb{\char`\}}
\def\PYZca{\char`\^}
\def\PYZam{\char`\&}
\def\PYZlt{\char`\<}
\def\PYZgt{\char`\>}
\def\PYZsh{\char`\#}
\def\PYZpc{\char`\%}
\def\PYZdl{\char`\$}
\def\PYZhy{\char`\-}
\def\PYZsq{\char`\'}
\def\PYZdq{\char`\"}
\def\PYZti{\char`\~}
% for compatibility with earlier versions
\def\PYZat{@}
\def\PYZlb{[}
\def\PYZrb{]}
\makeatother


    % For linebreaks inside Verbatim environment from package fancyvrb. 
    \makeatletter
        \newbox\Wrappedcontinuationbox 
        \newbox\Wrappedvisiblespacebox 
        \newcommand*\Wrappedvisiblespace {\textcolor{red}{\textvisiblespace}} 
        \newcommand*\Wrappedcontinuationsymbol {\textcolor{red}{\llap{\tiny$\m@th\hookrightarrow$}}} 
        \newcommand*\Wrappedcontinuationindent {3ex } 
        \newcommand*\Wrappedafterbreak {\kern\Wrappedcontinuationindent\copy\Wrappedcontinuationbox} 
        % Take advantage of the already applied Pygments mark-up to insert 
        % potential linebreaks for TeX processing. 
        %        {, <, #, %, $, ' and ": go to next line. 
        %        _, }, ^, &, >, - and ~: stay at end of broken line. 
        % Use of \textquotesingle for straight quote. 
        \newcommand*\Wrappedbreaksatspecials {% 
            \def\PYGZus{\discretionary{\char`\_}{\Wrappedafterbreak}{\char`\_}}% 
            \def\PYGZob{\discretionary{}{\Wrappedafterbreak\char`\{}{\char`\{}}% 
            \def\PYGZcb{\discretionary{\char`\}}{\Wrappedafterbreak}{\char`\}}}% 
            \def\PYGZca{\discretionary{\char`\^}{\Wrappedafterbreak}{\char`\^}}% 
            \def\PYGZam{\discretionary{\char`\&}{\Wrappedafterbreak}{\char`\&}}% 
            \def\PYGZlt{\discretionary{}{\Wrappedafterbreak\char`\<}{\char`\<}}% 
            \def\PYGZgt{\discretionary{\char`\>}{\Wrappedafterbreak}{\char`\>}}% 
            \def\PYGZsh{\discretionary{}{\Wrappedafterbreak\char`\#}{\char`\#}}% 
            \def\PYGZpc{\discretionary{}{\Wrappedafterbreak\char`\%}{\char`\%}}% 
            \def\PYGZdl{\discretionary{}{\Wrappedafterbreak\char`\$}{\char`\$}}% 
            \def\PYGZhy{\discretionary{\char`\-}{\Wrappedafterbreak}{\char`\-}}% 
            \def\PYGZsq{\discretionary{}{\Wrappedafterbreak\textquotesingle}{\textquotesingle}}% 
            \def\PYGZdq{\discretionary{}{\Wrappedafterbreak\char`\"}{\char`\"}}% 
            \def\PYGZti{\discretionary{\char`\~}{\Wrappedafterbreak}{\char`\~}}% 
        } 
        % Some characters . , ; ? ! / are not pygmentized. 
        % This macro makes them "active" and they will insert potential linebreaks 
        \newcommand*\Wrappedbreaksatpunct {% 
            \lccode`\~`\.\lowercase{\def~}{\discretionary{\hbox{\char`\.}}{\Wrappedafterbreak}{\hbox{\char`\.}}}% 
            \lccode`\~`\,\lowercase{\def~}{\discretionary{\hbox{\char`\,}}{\Wrappedafterbreak}{\hbox{\char`\,}}}% 
            \lccode`\~`\;\lowercase{\def~}{\discretionary{\hbox{\char`\;}}{\Wrappedafterbreak}{\hbox{\char`\;}}}% 
            \lccode`\~`\:\lowercase{\def~}{\discretionary{\hbox{\char`\:}}{\Wrappedafterbreak}{\hbox{\char`\:}}}% 
            \lccode`\~`\?\lowercase{\def~}{\discretionary{\hbox{\char`\?}}{\Wrappedafterbreak}{\hbox{\char`\?}}}% 
            \lccode`\~`\!\lowercase{\def~}{\discretionary{\hbox{\char`\!}}{\Wrappedafterbreak}{\hbox{\char`\!}}}% 
            \lccode`\~`\/\lowercase{\def~}{\discretionary{\hbox{\char`\/}}{\Wrappedafterbreak}{\hbox{\char`\/}}}% 
            \catcode`\.\active
            \catcode`\,\active 
            \catcode`\;\active
            \catcode`\:\active
            \catcode`\?\active
            \catcode`\!\active
            \catcode`\/\active 
            \lccode`\~`\~ 	
        }
    \makeatother

    \let\OriginalVerbatim=\Verbatim
    \makeatletter
    \renewcommand{\Verbatim}[1][1]{%
        %\parskip\z@skip
        \sbox\Wrappedcontinuationbox {\Wrappedcontinuationsymbol}%
        \sbox\Wrappedvisiblespacebox {\FV@SetupFont\Wrappedvisiblespace}%
        \def\FancyVerbFormatLine ##1{\hsize\linewidth
            \vtop{\raggedright\hyphenpenalty\z@\exhyphenpenalty\z@
                \doublehyphendemerits\z@\finalhyphendemerits\z@
                \strut ##1\strut}%
        }%
        % If the linebreak is at a space, the latter will be displayed as visible
        % space at end of first line, and a continuation symbol starts next line.
        % Stretch/shrink are however usually zero for typewriter font.
        \def\FV@Space {%
            \nobreak\hskip\z@ plus\fontdimen3\font minus\fontdimen4\font
            \discretionary{\copy\Wrappedvisiblespacebox}{\Wrappedafterbreak}
            {\kern\fontdimen2\font}%
        }%
        
        % Allow breaks at special characters using \PYG... macros.
        \Wrappedbreaksatspecials
        % Breaks at punctuation characters . , ; ? ! and / need catcode=\active 	
        \OriginalVerbatim[#1,codes*=\Wrappedbreaksatpunct]%
    }
    \makeatother

    % Exact colors from NB
    \definecolor{incolor}{HTML}{303F9F}
    \definecolor{outcolor}{HTML}{D84315}
    \definecolor{cellborder}{HTML}{CFCFCF}
    \definecolor{cellbackground}{HTML}{F7F7F7}
    
    % prompt
    \makeatletter
    \newcommand{\boxspacing}{\kern\kvtcb@left@rule\kern\kvtcb@boxsep}
    \makeatother
    \newcommand{\prompt}[4]{
        \ttfamily\llap{{\color{#2}[#3]:\hspace{3pt}#4}}\vspace{-\baselineskip}
    }
    

    
    % Prevent overflowing lines due to hard-to-break entities
    \sloppy 
    % Setup hyperref package
    \hypersetup{
      breaklinks=true,  % so long urls are correctly broken across lines
      colorlinks=true,
      urlcolor=urlcolor,
      linkcolor=linkcolor,
      citecolor=citecolor,
      }
    % Slightly bigger margins than the latex defaults
    
    \geometry{verbose,tmargin=1in,bmargin=1in,lmargin=1in,rmargin=1in}
    
    

\begin{document}
    
    \maketitle
    
    

    
    \hypertarget{option-pricing-and-prepare-for-risk}{%
\section{Option Pricing and Prepare for
Risk}\label{option-pricing-and-prepare-for-risk}}

\begin{itemize}
\tightlist
\item
  Uehara Takahiro
\item
  Tokyo Institute of Technology\\
\item
  Master Degree\\
\item
  takahiro.steadtler@gmail.com
\end{itemize}

    \begin{tcolorbox}[breakable, size=fbox, boxrule=1pt, pad at break*=1mm,colback=cellbackground, colframe=cellborder]
\prompt{In}{incolor}{26}{\boxspacing}
\begin{Verbatim}[commandchars=\\\{\}]
\PY{k+kn}{import} \PY{n+nn}{os}
\PY{k+kn}{import} \PY{n+nn}{numpy} \PY{k}{as} \PY{n+nn}{np}
\PY{k+kn}{import} \PY{n+nn}{pandas} \PY{k}{as} \PY{n+nn}{pd}
\PY{k+kn}{import} \PY{n+nn}{xgboost} \PY{k}{as} \PY{n+nn}{xgb}
\PY{k+kn}{import} \PY{n+nn}{seaborn} \PY{k}{as} \PY{n+nn}{sns}
\PY{k+kn}{from} \PY{n+nn}{xgboost} \PY{k+kn}{import} \PY{n}{DMatrix}
\PY{k+kn}{import} \PY{n+nn}{matplotlib}\PY{n+nn}{.}\PY{n+nn}{pyplot} \PY{k}{as} \PY{n+nn}{plt}
\PY{k+kn}{from} \PY{n+nn}{xgboost} \PY{k+kn}{import} \PY{n}{plot\PYZus{}importance}\PY{p}{,} \PY{n}{plot\PYZus{}tree}
\PY{k+kn}{from} \PY{n+nn}{sklearn}\PY{n+nn}{.}\PY{n+nn}{metrics} \PY{k+kn}{import} \PY{n}{mean\PYZus{}squared\PYZus{}error}
\PY{k+kn}{from} \PY{n+nn}{sklearn}\PY{n+nn}{.}\PY{n+nn}{preprocessing} \PY{k+kn}{import} \PY{n}{MinMaxScaler}
\PY{k+kn}{from} \PY{n+nn}{sklearn}\PY{n+nn}{.}\PY{n+nn}{model\PYZus{}selection} \PY{k+kn}{import} \PY{n}{train\PYZus{}test\PYZus{}split}\PY{p}{,} \PY{n}{GridSearchCV}
\PY{k+kn}{from} \PY{n+nn}{sklearn}\PY{n+nn}{.}\PY{n+nn}{metrics} \PY{k+kn}{import} \PY{n}{confusion\PYZus{}matrix}\PY{p}{,} \PY{n}{recall\PYZus{}score}
\PY{k+kn}{import} \PY{n+nn}{lightgbm} \PY{k}{as} \PY{n+nn}{lgb}
\PY{k+kn}{from} \PY{n+nn}{sklearn}\PY{n+nn}{.}\PY{n+nn}{metrics} \PY{k+kn}{import} \PY{n}{accuracy\PYZus{}score}
\PY{k+kn}{from} \PY{n+nn}{tqdm} \PY{k+kn}{import} \PY{n}{tqdm}
\PY{k+kn}{import} \PY{n+nn}{plotly} \PY{k}{as} \PY{n+nn}{py}
\PY{k+kn}{import} \PY{n+nn}{plotly}\PY{n+nn}{.}\PY{n+nn}{io} \PY{k}{as} \PY{n+nn}{pio}
\PY{k+kn}{import} \PY{n+nn}{plotly}\PY{n+nn}{.}\PY{n+nn}{graph\PYZus{}objects} \PY{k}{as} \PY{n+nn}{go}
\PY{k+kn}{from} \PY{n+nn}{plotly}\PY{n+nn}{.}\PY{n+nn}{subplots} \PY{k+kn}{import} \PY{n}{make\PYZus{}subplots}
\PY{k+kn}{from} \PY{n+nn}{plotly}\PY{n+nn}{.}\PY{n+nn}{offline} \PY{k+kn}{import} \PY{n}{download\PYZus{}plotlyjs}\PY{p}{,} \PY{n}{init\PYZus{}notebook\PYZus{}mode}\PY{p}{,} \PY{n}{plot}\PY{p}{,} \PY{n}{iplot}
\PY{k+kn}{from} \PY{n+nn}{warnings} \PY{k+kn}{import} \PY{n}{simplefilter}
\end{Verbatim}
\end{tcolorbox}

    \hypertarget{ready-const}{%
\subsection{Ready const}\label{ready-const}}

ready constant given by task mail

    \begin{tcolorbox}[breakable, size=fbox, boxrule=1pt, pad at break*=1mm,colback=cellbackground, colframe=cellborder]
\prompt{In}{incolor}{27}{\boxspacing}
\begin{Verbatim}[commandchars=\\\{\}]
\PY{c+c1}{\PYZsh{} current price of underlying asset}
\PY{n}{current\PYZus{}price} \PY{o}{=} \PY{l+m+mi}{100}
\PY{c+c1}{\PYZsh{} array of pair of probability and tommorow price}
\PY{c+c1}{\PYZsh{} lower\PYZus{}price must be lower than current price, higher\PYZus{}price must be higher than current price, }
\PY{n}{patterns} \PY{o}{=} \PY{p}{\PYZob{}}\PY{l+s+s1}{\PYZsq{}}\PY{l+s+s1}{lower\PYZus{}price}\PY{l+s+s1}{\PYZsq{}} \PY{p}{:} \PY{p}{\PYZob{}}\PY{l+s+s1}{\PYZsq{}}\PY{l+s+s1}{probability}\PY{l+s+s1}{\PYZsq{}}\PY{p}{:} \PY{l+m+mf}{0.5}\PY{p}{,} \PY{l+s+s1}{\PYZsq{}}\PY{l+s+s1}{tomorrow\PYZus{}rate}\PY{l+s+s1}{\PYZsq{}}\PY{p}{:} \PY{l+m+mf}{0.9}\PY{p}{\PYZcb{}}\PY{p}{,} \PY{l+s+s1}{\PYZsq{}}\PY{l+s+s1}{higher\PYZus{}price}\PY{l+s+s1}{\PYZsq{}} \PY{p}{:} \PY{p}{\PYZob{}}\PY{l+s+s1}{\PYZsq{}}\PY{l+s+s1}{probability}\PY{l+s+s1}{\PYZsq{}}\PY{p}{:} \PY{l+m+mf}{0.5}\PY{p}{,} \PY{l+s+s1}{\PYZsq{}}\PY{l+s+s1}{tomorrow\PYZus{}rate}\PY{l+s+s1}{\PYZsq{}}\PY{p}{:} \PY{l+m+mf}{1.2}\PY{p}{\PYZcb{}}\PY{p}{\PYZcb{}}
\PY{c+c1}{\PYZsh{} ten year risk free rate}
\PY{n}{ten\PYZus{}year\PYZus{}risk\PYZus{}free\PYZus{}rate} \PY{o}{=} \PY{l+m+mf}{0.0081}
\PY{c+c1}{\PYZsh{} one day risk free rate}
\PY{n}{one\PYZus{}day\PYZus{}risk\PYZus{}free\PYZus{}rate} \PY{o}{=} \PY{l+m+mf}{1.0081} \PY{o}{*}\PY{o}{*} \PY{p}{(}\PY{l+m+mi}{1}\PY{o}{/}\PY{l+m+mi}{3650}\PY{p}{)}
\PY{c+c1}{\PYZsh{} calc probability based on risk free rate and tomorrow rate}
\PY{n}{q} \PY{o}{=} \PY{p}{(}\PY{n}{one\PYZus{}day\PYZus{}risk\PYZus{}free\PYZus{}rate} \PY{o}{\PYZhy{}} \PY{n}{patterns}\PY{p}{[}\PY{l+s+s1}{\PYZsq{}}\PY{l+s+s1}{lower\PYZus{}price}\PY{l+s+s1}{\PYZsq{}}\PY{p}{]}\PY{p}{[}\PY{l+s+s1}{\PYZsq{}}\PY{l+s+s1}{tomorrow\PYZus{}rate}\PY{l+s+s1}{\PYZsq{}}\PY{p}{]}\PY{p}{)} \PY{o}{/} \PY{p}{(}\PY{n}{patterns}\PY{p}{[}\PY{l+s+s1}{\PYZsq{}}\PY{l+s+s1}{higher\PYZus{}price}\PY{l+s+s1}{\PYZsq{}}\PY{p}{]}\PY{p}{[}\PY{l+s+s1}{\PYZsq{}}\PY{l+s+s1}{tomorrow\PYZus{}rate}\PY{l+s+s1}{\PYZsq{}}\PY{p}{]} \PY{o}{\PYZhy{}} \PY{n}{patterns}\PY{p}{[}\PY{l+s+s1}{\PYZsq{}}\PY{l+s+s1}{lower\PYZus{}price}\PY{l+s+s1}{\PYZsq{}}\PY{p}{]}\PY{p}{[}\PY{l+s+s1}{\PYZsq{}}\PY{l+s+s1}{tomorrow\PYZus{}rate}\PY{l+s+s1}{\PYZsq{}}\PY{p}{]}\PY{p}{)}
\end{Verbatim}
\end{tcolorbox}

    \hypertarget{how-to-solve-task2}{%
\subsection{How to solve task2}\label{how-to-solve-task2}}

I would like to use binomial model to find the option premium
corresponding to the strike price.\\
In addition, I would like to use the same theory to construct a
portfolio to protect against risk.

    \hypertarget{functions-to-calculate-the-option-premium-based-on-the-strike-price}{%
\subsection{Functions to calculate the option premium based on the
strike
price}\label{functions-to-calculate-the-option-premium-based-on-the-strike-price}}

In this case, we define the variables as follows.\\
\$u : \$The first of the underlying asset price projections\\
\$d : \$The second of the underlying asset price projections\\
\$R : \$Risk-free interest rate\\
The price of the options if each state of \(u\) and \(d\) is realized is
as shown in the equation below.\\
\(C_u = max(uS - K, 0)\)\\
\(C_d = max(dS - K, 0)\)\\
The function named
``option\_premium\_depends\_on\_tomorrow\_rate\_and\_strike\_price'' at
the bottom represents the two expressions above.

If we know the probability of each state happening, we can find the
current option price as an expected value.\\
The reason we divide by R is to take into account the risk-free
assets.\\
We have expressed this in the equation below.\\
\(C = \frac{1}{R}[qC_u + (1-q)C_d]\)\\
The function named ``option\_premium\_based\_on\_probability''
represents the above equation.

    \begin{tcolorbox}[breakable, size=fbox, boxrule=1pt, pad at break*=1mm,colback=cellbackground, colframe=cellborder]
\prompt{In}{incolor}{28}{\boxspacing}
\begin{Verbatim}[commandchars=\\\{\}]
\PY{k}{def} \PY{n+nf}{option\PYZus{}premium\PYZus{}depends\PYZus{}on\PYZus{}tomorrow\PYZus{}rate\PYZus{}and\PYZus{}strike\PYZus{}price}\PY{p}{(}\PY{n}{strike\PYZus{}price}\PY{p}{,} \PY{n}{pattern}\PY{p}{)}\PY{p}{:}
    \PY{k}{return} \PY{n+nb}{max}\PY{p}{(}\PY{n}{pattern}\PY{p}{[}\PY{l+s+s1}{\PYZsq{}}\PY{l+s+s1}{tomorrow\PYZus{}rate}\PY{l+s+s1}{\PYZsq{}}\PY{p}{]} \PY{o}{*} \PY{n}{current\PYZus{}price} \PY{o}{\PYZhy{}} \PY{n}{strike\PYZus{}price}\PY{p}{,} \PY{l+m+mi}{0}\PY{p}{)}

\PY{k}{def} \PY{n+nf}{option\PYZus{}premium\PYZus{}based\PYZus{}on\PYZus{}probability}\PY{p}{(}\PY{n}{strike\PYZus{}price}\PY{p}{,} \PY{n}{probability} \PY{o}{=} \PY{n}{q}\PY{p}{)}\PY{p}{:}
    \PY{k}{return} \PY{p}{(}\PY{n}{probability} \PY{o}{*} \PY{n}{option\PYZus{}premium\PYZus{}depends\PYZus{}on\PYZus{}tomorrow\PYZus{}rate\PYZus{}and\PYZus{}strike\PYZus{}price}\PY{p}{(}\PY{n}{strike\PYZus{}price}\PY{p}{,} \PY{n}{patterns}\PY{p}{[}\PY{l+s+s1}{\PYZsq{}}\PY{l+s+s1}{higher\PYZus{}price}\PY{l+s+s1}{\PYZsq{}}\PY{p}{]}\PY{p}{)} \PY{o}{+} \PY{p}{(}\PY{l+m+mi}{1} \PY{o}{\PYZhy{}} \PY{n}{probability}\PY{p}{)} \PY{o}{*} \PY{n}{option\PYZus{}premium\PYZus{}depends\PYZus{}on\PYZus{}tomorrow\PYZus{}rate\PYZus{}and\PYZus{}strike\PYZus{}price}\PY{p}{(}\PY{n}{strike\PYZus{}price}\PY{p}{,} \PY{n}{patterns}\PY{p}{[}\PY{l+s+s1}{\PYZsq{}}\PY{l+s+s1}{lower\PYZus{}price}\PY{l+s+s1}{\PYZsq{}}\PY{p}{]}\PY{p}{)}\PY{p}{)} \PY{o}{/} \PY{n}{one\PYZus{}day\PYZus{}risk\PYZus{}free\PYZus{}rate}
\end{Verbatim}
\end{tcolorbox}

    \hypertarget{function-to-structure-the-portfolio-based-on-the-strike-price}{%
\subsection{Function to structure the portfolio based on the strike
price}\label{function-to-structure-the-portfolio-based-on-the-strike-price}}

I would like to replicate the option prices in the event of each of
\(u\) and \(d\), with a portfolio of risk-free assets and the underlying
assets.\\
The \(x\) represents the amount of the underlying asset should have.\\
The \(b\) represents the amount of the risk-free assets should have.\\
\(ux + Rb = C_u\)\\
\(dx + Rb = C_d\)\\
Solving this simultaneous equation, \(x\) and \(b\) are obtained as
follows.\\
\(x = \frac{C_u - C_d}{u - d}\)\\
\(b = \frac{uC_d - dC_u}{R(u - d)}\)\\
The function named ``doller\_to\_have\_stock'' at the bottom represents
\(x\) above.\\
The function named ``doller\_to\_have\_risk\_free'' at the bottom
represents \(b\) above.

    \begin{tcolorbox}[breakable, size=fbox, boxrule=1pt, pad at break*=1mm,colback=cellbackground, colframe=cellborder]
\prompt{In}{incolor}{29}{\boxspacing}
\begin{Verbatim}[commandchars=\\\{\}]
\PY{k}{def} \PY{n+nf}{doller\PYZus{}to\PYZus{}have\PYZus{}stock}\PY{p}{(}\PY{n}{strike\PYZus{}price}\PY{p}{)}\PY{p}{:}
    \PY{k}{return} \PY{p}{(}\PY{n}{option\PYZus{}premium\PYZus{}depends\PYZus{}on\PYZus{}tomorrow\PYZus{}rate\PYZus{}and\PYZus{}strike\PYZus{}price}\PY{p}{(}\PY{n}{strike\PYZus{}price}\PY{p}{,} \PY{n}{patterns}\PY{p}{[}\PY{l+s+s1}{\PYZsq{}}\PY{l+s+s1}{higher\PYZus{}price}\PY{l+s+s1}{\PYZsq{}}\PY{p}{]}\PY{p}{)} \PY{o}{\PYZhy{}} \PY{n}{option\PYZus{}premium\PYZus{}depends\PYZus{}on\PYZus{}tomorrow\PYZus{}rate\PYZus{}and\PYZus{}strike\PYZus{}price}\PY{p}{(}\PY{n}{strike\PYZus{}price}\PY{p}{,} \PY{n}{patterns}\PY{p}{[}\PY{l+s+s1}{\PYZsq{}}\PY{l+s+s1}{lower\PYZus{}price}\PY{l+s+s1}{\PYZsq{}}\PY{p}{]}\PY{p}{)}\PY{p}{)} \PY{o}{/} \PY{p}{(}\PY{n}{patterns}\PY{p}{[}\PY{l+s+s1}{\PYZsq{}}\PY{l+s+s1}{higher\PYZus{}price}\PY{l+s+s1}{\PYZsq{}}\PY{p}{]}\PY{p}{[}\PY{l+s+s1}{\PYZsq{}}\PY{l+s+s1}{tomorrow\PYZus{}rate}\PY{l+s+s1}{\PYZsq{}}\PY{p}{]} \PY{o}{\PYZhy{}} \PY{n}{patterns}\PY{p}{[}\PY{l+s+s1}{\PYZsq{}}\PY{l+s+s1}{lower\PYZus{}price}\PY{l+s+s1}{\PYZsq{}}\PY{p}{]}\PY{p}{[}\PY{l+s+s1}{\PYZsq{}}\PY{l+s+s1}{tomorrow\PYZus{}rate}\PY{l+s+s1}{\PYZsq{}}\PY{p}{]}\PY{p}{)}

\PY{k}{def} \PY{n+nf}{doller\PYZus{}to\PYZus{}have\PYZus{}risk\PYZus{}free}\PY{p}{(}\PY{n}{strike\PYZus{}price}\PY{p}{)}\PY{p}{:}
    \PY{k}{return} \PY{p}{(}\PY{n}{patterns}\PY{p}{[}\PY{l+s+s1}{\PYZsq{}}\PY{l+s+s1}{higher\PYZus{}price}\PY{l+s+s1}{\PYZsq{}}\PY{p}{]}\PY{p}{[}\PY{l+s+s1}{\PYZsq{}}\PY{l+s+s1}{tomorrow\PYZus{}rate}\PY{l+s+s1}{\PYZsq{}}\PY{p}{]} \PY{o}{*} \PY{n}{option\PYZus{}premium\PYZus{}depends\PYZus{}on\PYZus{}tomorrow\PYZus{}rate\PYZus{}and\PYZus{}strike\PYZus{}price}\PY{p}{(}\PY{n}{strike\PYZus{}price}\PY{p}{,} \PY{n}{patterns}\PY{p}{[}\PY{l+s+s1}{\PYZsq{}}\PY{l+s+s1}{lower\PYZus{}price}\PY{l+s+s1}{\PYZsq{}}\PY{p}{]}\PY{p}{)} \PY{o}{\PYZhy{}} \PY{n}{patterns}\PY{p}{[}\PY{l+s+s1}{\PYZsq{}}\PY{l+s+s1}{lower\PYZus{}price}\PY{l+s+s1}{\PYZsq{}}\PY{p}{]}\PY{p}{[}\PY{l+s+s1}{\PYZsq{}}\PY{l+s+s1}{tomorrow\PYZus{}rate}\PY{l+s+s1}{\PYZsq{}}\PY{p}{]} \PY{o}{*} \PY{n}{option\PYZus{}premium\PYZus{}depends\PYZus{}on\PYZus{}tomorrow\PYZus{}rate\PYZus{}and\PYZus{}strike\PYZus{}price}\PY{p}{(}\PY{n}{strike\PYZus{}price}\PY{p}{,} \PY{n}{patterns}\PY{p}{[}\PY{l+s+s1}{\PYZsq{}}\PY{l+s+s1}{higher\PYZus{}price}\PY{l+s+s1}{\PYZsq{}}\PY{p}{]}\PY{p}{)}\PY{p}{)} \PY{o}{/} \PY{p}{(}\PY{n}{one\PYZus{}day\PYZus{}risk\PYZus{}free\PYZus{}rate} \PY{o}{*} \PY{p}{(}\PY{n}{patterns}\PY{p}{[}\PY{l+s+s1}{\PYZsq{}}\PY{l+s+s1}{higher\PYZus{}price}\PY{l+s+s1}{\PYZsq{}}\PY{p}{]}\PY{p}{[}\PY{l+s+s1}{\PYZsq{}}\PY{l+s+s1}{tomorrow\PYZus{}rate}\PY{l+s+s1}{\PYZsq{}}\PY{p}{]} \PY{o}{\PYZhy{}} \PY{n}{patterns}\PY{p}{[}\PY{l+s+s1}{\PYZsq{}}\PY{l+s+s1}{lower\PYZus{}price}\PY{l+s+s1}{\PYZsq{}}\PY{p}{]}\PY{p}{[}\PY{l+s+s1}{\PYZsq{}}\PY{l+s+s1}{tomorrow\PYZus{}rate}\PY{l+s+s1}{\PYZsq{}}\PY{p}{]}\PY{p}{)}\PY{p}{)}
\end{Verbatim}
\end{tcolorbox}

    \hypertarget{outputs-values-according-to-the-strike-price}{%
\subsection{Outputs values according to the strike
price}\label{outputs-values-according-to-the-strike-price}}

I output each value by varying the strike price from 0 to 200.\\
- ``option\_premium represents'' represents an option premium that
requires the customer to pay(The probability of \(u\) and \(d\) are
based on the above q).\\
- ``option\_premium\_equal\_probability'' represents an option premium
that requires the customer to pay(The probability of \(u\) and \(d\) are
equal.).\\
- ``stock\_doller\_in\_portfolio'' represents how much of the underlying
asset we should hold when we sell one option.\\
- ``risk\_free\_doller\_in\_portfolio'' represents how much of a
risk-free asset we should hold when we sell one option.

    \begin{tcolorbox}[breakable, size=fbox, boxrule=1pt, pad at break*=1mm,colback=cellbackground, colframe=cellborder]
\prompt{In}{incolor}{32}{\boxspacing}
\begin{Verbatim}[commandchars=\\\{\}]
\PY{n}{results} \PY{o}{=} \PY{n}{pd}\PY{o}{.}\PY{n}{DataFrame}\PY{p}{(}\PY{n}{columns} \PY{o}{=} \PY{p}{[}\PY{l+s+s1}{\PYZsq{}}\PY{l+s+s1}{strike\PYZus{}price}\PY{l+s+s1}{\PYZsq{}}\PY{p}{,} \PY{l+s+s1}{\PYZsq{}}\PY{l+s+s1}{option\PYZus{}premium}\PY{l+s+s1}{\PYZsq{}}\PY{p}{,} \PY{l+s+s1}{\PYZsq{}}\PY{l+s+s1}{option\PYZus{}premium\PYZus{}equal\PYZus{}probability}\PY{l+s+s1}{\PYZsq{}}\PY{p}{,} \PY{l+s+s1}{\PYZsq{}}\PY{l+s+s1}{stock\PYZus{}doller\PYZus{}in\PYZus{}portfolio}\PY{l+s+s1}{\PYZsq{}}\PY{p}{,} \PY{l+s+s1}{\PYZsq{}}\PY{l+s+s1}{risk\PYZus{}free\PYZus{}doller\PYZus{}in\PYZus{}portfolio}\PY{l+s+s1}{\PYZsq{}}\PY{p}{]}\PY{p}{)}
\PY{n}{strike\PYZus{}prices} \PY{o}{=} \PY{p}{[}\PY{p}{]}
\PY{n}{option\PYZus{}premiums} \PY{o}{=} \PY{p}{[}\PY{p}{]}
\PY{n}{stock\PYZus{}dollers\PYZus{}in\PYZus{}portfolio} \PY{o}{=} \PY{p}{[}\PY{p}{]}
\PY{n}{option\PYZus{}premiums\PYZus{}equal\PYZus{}probability} \PY{o}{=} \PY{p}{[}\PY{p}{]}
\PY{n}{risk\PYZus{}free\PYZus{}dollers\PYZus{}in\PYZus{}portfolio} \PY{o}{=} \PY{p}{[}\PY{p}{]}
\PY{k}{for} \PY{n}{strike\PYZus{}price} \PY{o+ow}{in} \PY{n+nb}{range}\PY{p}{(}\PY{n+nb}{max}\PY{p}{(}\PY{n}{current\PYZus{}price} \PY{o}{\PYZhy{}} \PY{l+m+mi}{1000}\PY{p}{,} \PY{l+m+mi}{0}\PY{p}{)}\PY{p}{,} \PY{n}{current\PYZus{}price} \PY{o}{*} \PY{l+m+mi}{2}\PY{p}{)}\PY{p}{:}
    \PY{n}{strike\PYZus{}prices}\PY{o}{.}\PY{n}{append}\PY{p}{(}\PY{n}{strike\PYZus{}price}\PY{p}{)}
    \PY{n}{option\PYZus{}premiums}\PY{o}{.}\PY{n}{append}\PY{p}{(}\PY{n}{option\PYZus{}premium\PYZus{}based\PYZus{}on\PYZus{}probability}\PY{p}{(}\PY{n}{strike\PYZus{}price}\PY{p}{,} \PY{n}{q}\PY{p}{)}\PY{p}{)}
    \PY{n}{option\PYZus{}premiums\PYZus{}equal\PYZus{}probability}\PY{o}{.}\PY{n}{append}\PY{p}{(}\PY{n}{option\PYZus{}premium\PYZus{}based\PYZus{}on\PYZus{}probability}\PY{p}{(}\PY{n}{strike\PYZus{}price}\PY{p}{,} \PY{l+m+mf}{0.5}\PY{p}{)}\PY{p}{)}
    \PY{n}{stock\PYZus{}dollers\PYZus{}in\PYZus{}portfolio}\PY{o}{.}\PY{n}{append}\PY{p}{(}\PY{n}{doller\PYZus{}to\PYZus{}have\PYZus{}stock}\PY{p}{(}\PY{n}{strike\PYZus{}price}\PY{p}{)}\PY{p}{)}
    \PY{n}{risk\PYZus{}free\PYZus{}dollers\PYZus{}in\PYZus{}portfolio}\PY{o}{.}\PY{n}{append}\PY{p}{(}\PY{n}{doller\PYZus{}to\PYZus{}have\PYZus{}risk\PYZus{}free}\PY{p}{(}\PY{n}{strike\PYZus{}price}\PY{p}{)}\PY{p}{)}   
\PY{n}{results}\PY{p}{[}\PY{l+s+s1}{\PYZsq{}}\PY{l+s+s1}{strike\PYZus{}price}\PY{l+s+s1}{\PYZsq{}}\PY{p}{]} \PY{o}{=} \PY{n}{strike\PYZus{}prices}
\PY{n}{results}\PY{p}{[}\PY{l+s+s1}{\PYZsq{}}\PY{l+s+s1}{option\PYZus{}premium}\PY{l+s+s1}{\PYZsq{}}\PY{p}{]} \PY{o}{=} \PY{n}{option\PYZus{}premiums}
\PY{n}{results}\PY{p}{[}\PY{l+s+s1}{\PYZsq{}}\PY{l+s+s1}{option\PYZus{}premium\PYZus{}equal\PYZus{}probability}\PY{l+s+s1}{\PYZsq{}}\PY{p}{]} \PY{o}{=} \PY{n}{option\PYZus{}premiums\PYZus{}equal\PYZus{}probability}
\PY{n}{results}\PY{p}{[}\PY{l+s+s1}{\PYZsq{}}\PY{l+s+s1}{stock\PYZus{}doller\PYZus{}in\PYZus{}portfolio}\PY{l+s+s1}{\PYZsq{}}\PY{p}{]} \PY{o}{=} \PY{n}{stock\PYZus{}dollers\PYZus{}in\PYZus{}portfolio}
\PY{n}{results}\PY{p}{[}\PY{l+s+s1}{\PYZsq{}}\PY{l+s+s1}{risk\PYZus{}free\PYZus{}doller\PYZus{}in\PYZus{}portfolio}\PY{l+s+s1}{\PYZsq{}}\PY{p}{]} \PY{o}{=} \PY{n}{risk\PYZus{}free\PYZus{}dollers\PYZus{}in\PYZus{}portfolio}
\PY{n}{results} \PY{o}{=} \PY{n}{results}\PY{o}{.}\PY{n}{set\PYZus{}index}\PY{p}{(}\PY{l+s+s1}{\PYZsq{}}\PY{l+s+s1}{strike\PYZus{}price}\PY{l+s+s1}{\PYZsq{}}\PY{p}{)}
\PY{n}{results}\PY{o}{.}\PY{n}{plot}\PY{p}{(}\PY{p}{)}
\end{Verbatim}
\end{tcolorbox}

            \begin{tcolorbox}[breakable, size=fbox, boxrule=.5pt, pad at break*=1mm, opacityfill=0]
\prompt{Out}{outcolor}{32}{\boxspacing}
\begin{Verbatim}[commandchars=\\\{\}]
<matplotlib.axes.\_subplots.AxesSubplot at 0x7fc8fe1b3d10>
\end{Verbatim}
\end{tcolorbox}
        
    \begin{center}
    \adjustimage{max size={0.9\linewidth}{0.9\paperheight}}{option-pricing_files/option-pricing_10_1.png}
    \end{center}
    { \hspace*{\fill} \\}
    
    \hypertarget{answer}{%
\subsection{Answer}\label{answer}}

For ``task2'', I have to sell 1000 options, so I have to multiply the
value in results by 1000.\\
For example, if I sell the right to buy stock A for \$105 to a customer,
it would look like this below.\\
(If it is negative, it indicates that we are taking out a loan.)

    \begin{tcolorbox}[breakable, size=fbox, boxrule=1pt, pad at break*=1mm,colback=cellbackground, colframe=cellborder]
\prompt{In}{incolor}{43}{\boxspacing}
\begin{Verbatim}[commandchars=\\\{\}]
\PY{n}{results}\PY{p}{[}\PY{l+m+mi}{105}\PY{p}{:}\PY{l+m+mi}{106}\PY{p}{]} \PY{o}{*} \PY{l+m+mi}{1000}
\end{Verbatim}
\end{tcolorbox}

            \begin{tcolorbox}[breakable, size=fbox, boxrule=.5pt, pad at break*=1mm, opacityfill=0]
\prompt{Out}{outcolor}{43}{\boxspacing}
\begin{Verbatim}[commandchars=\\\{\}]
              option\_price  option\_price\_equal\_probability  \textbackslash{}
strike\_price
105            5000.099461                     7499.983423

              stock\_doller\_in\_portfolio  risk\_free\_doller\_in\_portfolio
strike\_price
105                             50000.0                  -44999.900539
\end{Verbatim}
\end{tcolorbox}
        

    % Add a bibliography block to the postdoc
    
    
    
\end{document}
